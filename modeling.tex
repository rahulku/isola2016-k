
\section{Observed concepts in modeling}

\subsection{Formal methods}

Early work on formal methods include the work of John Mclarthy, 
Robert Floyd (Assigning Meanings to Programs), Edsger Dijkstra (A 
Discipline of Programming), Tony Hoare (An Axiomatic Basis for 
Computer Programming), and Dana Scott and Christopher Strachey 
(Towards a Mathematical Semantics for Computer Languages), to 
mention a few. These ideas were theoretic in nature and deeply 
influential. They brought us the ideas of  annotating programs with 
assertions, such as pre- and post-conditions, and invariants, 
correct by construction development (refinement), and giving 
semantics to programming languages (denotational semantics). 

These ideas were subsequently the basis for several, what we could 
call second generation, formal specification languages such as VDM, 
RAISE, Z, TLA, and CIP. Each of these languages were full 
specification languages, with rich type systems and detailed rules 
(grammars) for what constituted a valid specification. These 
languages were ahead of their time wrt. language constructs in the 
sense that many of the language features found in these languages 
slowly are finding their way into modern programming languages of 
today. 

The VDM language for example was a wide-spectrum specification 
language offering a combination of high-level specification 
constructs and low level programming constructs. The methodology 
consists in part, as in CIP, of refining a high-level specification 
into a program in a stepwise manner. The language offered concepts 
such as the combination of imperative (procedural and later object 
oriented in VDM++) and functional programming; exceptions; 
algebraic data types and pattern matching; functions as values and 
lambda abstractions; built-in collection types such as sets, lists 
and maps, with mathematical notation for creating values of these 
types, such as for example set comprehension; design-by-contract 
through pre- and post conditions and invariants; predicate subtypes 
(so one for example can define natural numbers as a subset of the 
integers);  and predicate logic including universal and existential 
quantification over any type as Boolean expressions.  VDM and Z are 
so-called model-oriented specification languages, meaning that a 
specification is an example model of the desired system. This means 
that such specifications are somewhat close to high-level programs. 
This is in contrast to so-called property oriented specification 
languages, such as OBJ.  

A different branch of formal methods include theorem proving and 
model checking. In theorem proving we have seen specification 
languages, which resemble functional programming languages, 
including for example ACL, PVS and Coq. In model checking early 
work focused on modeling notations somewhat removed from 
programming languages. However, recent research has focused on 
software model checking, where the target of model checking is 
code, as for example seen in the Java PathFinder model checker, 
JPF. JPF was created due to the observation that a powerful 
programming language might be a better modeling language than the, 
at the time existing, model checker input languages. Today’s 
efforts in model checking furthermore include numerous efforts in 
model checking of C programs.

As can be seen from the above discussion, formal specification 
languages have for a long time been flirting with programming 
language like notations, and vice versa. However, the two classes of languages have by tradition been considered as belonging to 
strictly separate categories. VDM for example was always, and still 
is, considered a specification language, albeit with code 
generation capabilities. It has never, in spite of the possibility, 
been considered (named) a programming language, which one may 
consider being as one of the reasons it is not more wide spread. 
Writing specifications in VDM and generating code in Java, for 
example, has not become popular. Programmers feel uncomfortable 
working with two languages (a specification language and  a 
programming language) when the two languages are too similar. This 
is an argument for merging the concepts into a  specification, 
design and implementation language.

\subsection{Model-based engineering}

Model-based engineering includes modeling frameworks that are 
usually visual/graphical of nature. One of the main contributions 
in this field is UML for software development, and its derivatives, 
such as SysML for systems development. The graphical nature of the 
UML family of languages has caused it to become rather popular and 
wide-spread in engineering communities. Engineers are at first 
encounter more willing to work with graphical notations, such as 
class diagrams and state machines, than they are working with sets, 
lists and maps and function definitions. It seems clearly more 
accepted than formal methods as described in the previous section. 
At JPL for example, there are three people working with formal 
methods and over hundred working with UML/SysML technology. 

One of the important notations in UML/SysML is class diagrams. 
Class diagrams are – just like E/R-diagrams – really a simple way 
of defining data, an alternative to working with sets, lists and 
maps as found in VDM and modern programming languages. For example, 
to state that a person can own zero or more cars one draws a box 
for Person and a box for Car and draws a line between them. It is 
an idea that quickly be picked up by a systems engineer, quicker 
than learn to program with sets. Another notation is that of state 
machines, a concept that strangely enough has not found its way 
into programming languages, in spite of its usefulness in 
especially embedded programming. UML/SysML also focuses to some 
extent on requirements, a concept that usually is not embedded as a 
first class object in programming. It would be interesting to see 
requirements as part of programming.

The above observations are rather positive. However, UML/SysML are 
very complex and weakly defined formalisms. The combined syntax for 
all UML for example corresponds to the sum of approximately 20 
programming languages (approximately 17,000 lines of abstract 
syntax, a programming language can normally be defined in between 
500 and 2500 lines of grammar rules). The UML/SysML standards are 
long and complex documents. The connection between models and code 
is fragile, relying on the correctness of translators from for 
example UML state machines to code. 
