
%%%%%%%%%%%%%%%%%%%%%%%%%%%%%%%%%%%%%%%%%%%%%
% THIS IS THE ORIGINAL WHITE PAPER.         %
% DO NOT EDIT THIS FILE.                    %
% MAKE THE APPROPRIATE SECTIONS (FILES)     %
% AND COPY OVER AND EDIT THERE.             %
%%%%%%%%%%%%%%%%%%%%%%%%%%%%%%%%%%%%%%%%%%%%%

\section{Introduction}

Over the last several decades we have observed the development of a 
large collection of specification and modeling languages and 
associated method methodologies, and tools. Their purpose is to 
support modeling of requirements and high-level designs before 
programming is initiated. Agile approaches advocate to avoid 
explicit modeling entirely and suggest to go directly to coding. 
Other approaches advocate avoiding manual “coding” in a programming 
language entirely and suggest instead the generation of code 
directly from the models. This way modeling languages replace 
programming languages.  We can divide these modeling languages into 
formal specification languages (formal methods), usually focusing 
on textual languages based on mathematical logic and set theory, 
and associated proof tools (theorem provers, model checkers, etc.), 
and on the other hand model-based engineering languages (UML, 
SysML, Modelica, Mathematica, …), focusing more on design, code 
generation and simulation. Many of these modeling languages have 
similarities with programming languages.

In parallel, and frankly seemingly independent, we have seen the 
development of numerous new programming languages. Few languages 
have had the success of C, which still today is the main 
programming language for embedded systems. The success is so 
outstanding that nearly no progress wrt. praxis has been made in 
this domain (embedded programming) since the 1970ties.  In 
application programming a collection of new languages came ago such 
as Ada, C++, Eiffel etc. At the same time we have seen several 
high-level languages appear targeting the softer side of software 
engineering (such as web-programming, user interfaces, scripting), 
including languages such as Java, JavaScript, Ruby, Python and 
Scala.  More academic languages include Haskell and the ML family, 
including OCaml.

In this white paper we will give a brief overview of some of the, 
in our view, important developments in modeling and programming 
languages. Subsequently we will propose en effort to develop a 
specification, design and implementation language that combines 
modeling and programming.

\section{Observed concepts in modeling}

\subsection{Formal methods}

Early work on formal methods include the work of John Mclarthy, 
Robert Floyd (Assigning Meanings to Programs), Edsger Dijkstra (A 
Discipline of Programming), Tony Hoare (An Axiomatic Basis for 
Computer Programming), and Dana Scott and Christopher Strachey 
(Towards a Mathematical Semantics for Computer Languages), to 
mention a few. These ideas were theoretic in nature and deeply 
influential. They brought us the ideas of  annotating programs with 
assertions, such as pre- and post-conditions, and invariants, 
correct by construction development (refinement), and giving 
semantics to programming languages (denotational semantics). 

These ideas were subsequently the basis for several, what we could 
call second generation, formal specification languages such as VDM, 
RAISE, Z, TLA, and CIP. Each of these languages were full 
specification languages, with rich type systems and detailed rules 
(grammars) for what constituted a valid specification. These 
languages were ahead of their time wrt. language constructs in the 
sense that many of the language features found in these languages 
slowly are finding their way into modern programming languages of 
today. 

The VDM language for example was a wide-spectrum specification 
language offering a combination of high-level specification 
constructs and low level programming constructs. The methodology 
consists in part, as in CIP, of refining a high-level specification 
into a program in a stepwise manner. The language offered concepts 
such as the combination of imperative (procedural and later object 
oriented in VDM++) and functional programming; exceptions; 
algebraic data types and pattern matching; functions as values and 
lambda abstractions; built-in collection types such as sets, lists 
and maps, with mathematical notation for creating values of these 
types, such as for example set comprehension; design-by-contract 
through pre- and post conditions and invariants; predicate subtypes 
(so one for example can define natural numbers as a subset of the 
integers);  and predicate logic including universal and existential 
quantification over any type as Boolean expressions.  VDM and Z are 
so-called model-oriented specification languages, meaning that a 
specification is an example model of the desired system. This means 
that such specifications are somewhat close to high-level programs. 
This is in contrast to so-called property oriented specification 
languages, such as OBJ.  

A different branch of formal methods include theorem proving and 
model checking. In theorem proving we have seen specification 
languages, which resemble functional programming languages, 
including for example ACL, PVS and Coq. In model checking early 
work focused on modeling notations somewhat removed from 
programming languages. However, recent research has focused on 
software model checking, where the target of model checking is 
code, as for example seen in the Java PathFinder model checker, 
JPF. JPF was created due to the observation that a powerful 
programming language might be a better modeling language than the, 
at the time existing, model checker input languages. Today’s 
efforts in model checking furthermore include numerous efforts in 
model checking of C programs.

As can be seen from the above discussion, formal specification 
languages have for a long time been flirting with programming 
language like notations, and vice versa. However, the two classes of languages have by tradition been considered as belonging to 
strictly separate categories. VDM for example was always, and still 
is, considered a specification language, albeit with code 
generation capabilities. It has never, in spite of the possibility, 
been considered (named) a programming language, which one may 
consider being as one of the reasons it is not more wide spread. 
Writing specifications in VDM and generating code in Java, for 
example, has not become popular. Programmers feel uncomfortable 
working with two languages (a specification language and  a 
programming language) when the two languages are too similar. This 
is an argument for merging the concepts into a  specification, 
design and implementation language.

\subsection{Model-based engineering}

Model-based engineering includes modeling frameworks that are 
usually visual/graphical of nature. One of the main contributions 
in this field is UML for software development, and its derivatives, 
such as SysML for systems development. The graphical nature of the 
UML family of languages has caused it to become rather popular and 
wide-spread in engineering communities. Engineers are at first 
encounter more willing to work with graphical notations, such as 
class diagrams and state machines, than they are working with sets, 
lists and maps and function definitions. It seems clearly more 
accepted than formal methods as described in the previous section. 
At JPL for example, there are three people working with formal 
methods and over hundred working with UML/SysML technology. 

One of the important notations in UML/SysML is class diagrams. 
Class diagrams are – just like E/R-diagrams – really a simple way 
of defining data, an alternative to working with sets, lists and 
maps as found in VDM and modern programming languages. For example, 
to state that a person can own zero or more cars one draws a box 
for Person and a box for Car and draws a line between them. It is 
an idea that quickly be picked up by a systems engineer, quicker 
than learn to program with sets. Another notation is that of state 
machines, a concept that strangely enough has not found its way 
into programming languages, in spite of its usefulness in 
especially embedded programming. UML/SysML also focuses to some 
extent on requirements, a concept that usually is not embedded as a 
first class object in programming. It would be interesting to see 
requirements as part of programming.

The above observations are rather positive. However, UML/SysML are 
very complex and weakly defined formalisms. The combined syntax for 
all UML for example corresponds to the sum of approximately 20 
programming languages (approximately 17,000 lines of abstract 
syntax, a programming language can normally be defined in between 
500 and 2500 lines of grammar rules). The UML/SysML standards are 
long and complex documents. The connection between models and code 
is fragile, relying on the correctness of translators from for 
example UML state machines to code. 


\section{Observed concepts in programming}

Several new programming languages have emerged over the last 
decade, which include abstraction mechanisms known from the formal 
specification languages mentioned above. Such languages include 
Eiffel, Java, Python, Scala, Fortress,  C\#, Spec\#, F\#,  Dafny, D, 
RUST, Swift, Go, Agda, and SPARK.  Some languages support design-
by-contract with pre-post conditions, and in some cases with 
invariants. These languages  include for example Eiffel, Spec\#, 
Dafny, SPARK, and to some limited extent Scala. Java supports 
contracts through JML, which, however, is not integrated with Java, 
but an add-on comment language (JML specifications are comments in 
a Java program). Most of the languages above support abstract 
collections such as sets, lists and maps. It is interesting to 
observe that SUN’s Fortress language (which unfortunately was not 
finished by the designers) supports a mathematical notation for 
collections very similar to VDM. The Dafny language is interesting 
since it is developed specifically with specification and 
verification in mind.

A trend on the rise is likely the combination of object oriented 
and functional programming, as seen in perhaps most prominently 
Scala, but also in the earlier Python, and now in Java which got 
closures in version 1.8.  Ocaml is a similar earlier attempt to 
integrate object oriented and functional programming, although in a 
layered manner, and not integrated with the standard module system. 
Some interesting new directions of research include dependent types 
as found in Agda (to some extent related to predicate subtypes in 
VDM) and session types. Session types are temporal patterns that 
can be checked at compile time. They are much related to temporal 
logic as used within the formal methods community to express 
properties of concurrent programs. At the same time there are also 
attempts to make more conservative moves away from C, but without 
losing too much efficiency. Examples include the languages D and 
RUST.

\section{Requirements for a programming language}

A specification, design and implementation language will have to 
serve quite different goals. First of all, it has to represent the 
concepts of the application domain at an adequate level of 
abstraction such that the specialities of the applications domains 
are directly represented and not covered by awkward implementation 
concepts. This may for example include support for user-defined 
extensions of the language with domain-specific languages (DSLs). 
Second it has to address the structuring of algorithms and data 
structures in a way such that programs stay understandable, modular 
and support the most important methods of structured program 
development. And finally it has to allow addressing specific 
implementation properties of execution machines including their 
operating systems, such that it can be controlled how the 
implementation uses resources and exploits the possibilities of the 
execution platform and its hardware. 

These three different goals are clearly in some contradiction. 
Nevertheless in a piece of software all three goals have to be 
addressed. What we would like to have is a specification, design, 
and implementation language, which represents the concepts of the 
application domain as adequate as possible, which allows at the 
same time to structure the software in a readable and manageable 
form, and which allows addressing of particular execution concepts 
on the execution platform and the machine to the extent needed. 

An obvious problem here of course is to what extent then the 
particular application domain influences the programming languages, 
and to what extent this is true for the execution platform and also 
for the different forms of structured design concepts. In this 
section we shall try to outline what we see as the desired 
requirements of a programming language. 


\subsection{Target domain}

We can observe three major domains of interest, namely modeling; 
programming of non-critical systems, such as web applications, 
including scripting; and finally programming of embedded/cyber-
physical systems. It is clear that these three domains till date 
have been addressed by different communities and different 
languages, as outlined above. Our goal is to address the three 
domains in one language. Such a goal usually provokes a reactions 
which can be summarized as
“a silver bullet does not exist” and “each problem requires its own 
solution”. However, we do question this constraining view based on 
the observation that all of the languages above share a big set of 
language constructs.

\subsection{Support for modeling}

This item is less well defined, and generally means support for 
modeling and understanding a problem in addition to programming the 
solution. This is about combining modeling and programming into one 
language.  No separate UML models etc.  It includes design-by-
contract as we know it, including pre- and post-conditions, as well 
as class invariants. Such can for example be found in Eiffel as 
well as in SPARK. However, we believe that it can be carried 
further to for example include such topics as temporal logic, 
program monitoring, program visualization, including diagramming of 
static structure as well as dynamic behavior, as built in concepts, 
and of course verification to the point where it is practical, unit 
testing built in as in Fortress, etc.

\subsection{Design and architecture: programming in the large}

Large programs have to be structured. They have to be structured on 
one hand in independent or at least rather independent pieces that 
can be reused, independently changed, translated and executed on 
different hardware, such that a flexible deployment is achieved. 
They have to offer appropriate techniques for encapsulation and 
parameterisation. This structuring may also address issues of 
execution such as deployment and parallel execution.

\subsubsection{Components, modularity and encapsulation}

What is needed, in particular, is an appropriate notion of 
component as a unit of modularity and encapsulation. Such concepts 
exist in a lot of programming languages. However since most of the 
programming languages we are using today are inherently sequential, 
an independent deployment and execution model often is not directly 
achieved.

\subsubsection{Variability}

A key to efficient software evolution is the identification of 
components that can be used and reused at several places in a 
program. This requires a sufficient amount of variability. If such 
variability cannot be achieved then code cannot be reused at many 
places and as a result we have to form clones, meaning similar 
pieces of code with just small differences, such that they can be 
used at the individual places. Another issue for variability is the 
usage of different variations of software in the context of 
software families. Variability is an important concept, which is 
not very much supported explicitly by nowadays programming 
languages.

\subsection{Support for high-level programming}

The specification, design, and implementation notation has to be 
sufficiently abstract. Many formalisms used and suggested for that 
including a lot of the programming languages force the programmers 
to write too many details enforcing a particular style, which is 
related to a way to describe algorithms. Therefore the resulting 
programs get very long and more difficult to understand. The key 
question is how to provide programming concepts that are 
expressive, understandable, and do not enforce the explicit 
formulation of a lot of details due to a particular algorithmic 
style.


\subsubsection{Merging object-oriented and functional programming}

The elegance and the implicitly of functional programs have been 
praised many times. Nevertheless they never had an absolute 
breakthrough. In contrast, object oriented programming languages, 
which in particular address encapsulation and reuse were very 
successful. They provide entities of implementation called classes, 
which at the same time are able to present concepts in the 
application domain and units of execution. However, for all 
nowadays object oriented programming languages there are a number 
of properties, which do not allow using them in the required 
universal modelling style. One reason for that is that object 
oriented programming languages are inherently sequential due to 
their remote procedure call concept. All attempts to provide 
parallel execution models such as threads make things ugly and very 
complex. Therefore a good idea would be to use many of the good 
ideas in object oriented and functional programming and bringing 
them together in powerful generalizations. A language such as Scala 
has made this attempt, and even early versions of LISP had this 
(CLOS). Functional programming means for example functions as 
values (lambda abstractions) and pattern matching, and of course 
reliance on recursion. Functional programming is by some considered 
the best approach to use multi-core systems due to no shared state 
updates.

\subsubsection{Built-in collections}

Perhaps specialized notation for these as in Fortress (very similar 
to VDM), or as libraries. Easy ways of iterating through 
collections – to avoid indexing problems for example. Support for 
parallel computation over such.

\subsubsection{Domain-specific data modeling}

A key to programming is to capture the relevant concept of the 
application domain and to present them in the specification design 
and implementation notation. This is exactly where UML and also 
SysML are quite successful. They provide a number of concepts which 
originally were created in the area of programming and good enough 
to allow presenting quite a number of application domain issues. A 
typical examples are class style diagrams, which at a level of 
programming are describing more or less architectures in terms of 
classes and how they are connected, but can also be used as 
possibilities to describe data models and finally ontologies as we 
find them under the heading of meta models. In any case it is 
important to support powerful modelling approaches in the 
specification, design and implementation notation.

\subsubsection{Typing and physical dimensions}

We believe a language should be largely statically typed. It can 
potentially allow for going type-less in clearly defined regions, 
in case such makes modeling and programming easier. Scripting 
languages are popular, in part because they are type-less. It would 
be interesting to see if one could allow both approaches to be used 
within the same language. Otherwise decades of experience in strong 
type systems should of course be harvested, including more recent 
topics such as dependent types, session types, and units.

\subsection{Support for low-level programming}

Embedded programing these days often means: no dynamic memory 
allocation after initialization, no garbage collection, some 
knowledge of memory layout, even to the point where computation 
with addresses is used to improve speed. This again means use of 
low level programming languages such as C. C, however, allows for 
memory errors and makes programmers less effective as they would 
otherwise be were they allowed to program in higher-level 
languages. We need to satisfy the needs encountered by typical C 
programmers, including offering comparable speed and memory 
control. This includes support for hardware control.

\subsection{Concurrency}

Concurrency is an essential part of modern programming, especially 
considering the emergence of multi-core computers. However, 
concurrency is important at the modeling level as well, where it 
can serve as a natural way to describe interacting agents. 
Important concepts include agent systems, message passing based 
communication, parallel data structures (programming concurrent 
without knowing it), and distributed programming.

\subsection{Execution model}

The programming notation has to allow to control execution aspects. 
Often in today’s practice, extensions are introduced that allow 
controlling execution platforms. This is important in order to 
provide programs that are efficient. On the other hand, it is very 
dangerous since it mixes up domain specific concepts,  data 
modelling, and algorithms on one hand, and issues of specific 
execution concepts of the execution platform. Such a mix also makes 
it very difficult to port and migrate software. A promising 
approach could be that the specification, design, and 
implementation notation provides possibilities to target a specific 
execution platform by separate profiles that are in addition to the 
description of the domain specific concepts in the algorithms. This 
idea could be applied for time, concurrency, deployment and 
distribution.

\subsection{Analysis}

A key concept in a combined modeling and programming environment is 
the support for advanced analysis of models/programs, including, 
but also beyond, what is normally supported in standard programming 
environments. This ranges from basic built-in support for unit 
testing, over advanced testing capabilities, including test input 
generation and monitoring, to concepts such as static analysis, 
model checking, theorem proving and symbolic execution. A core 
requirement, however, must be the practicality of these solutions. 
The main emphasis should be put on automation. The average user 
should be able to benefit from automated verification, without 
having to do manual proofs. However, support for manual theorem 
proving should also be possible, for example for core critical 
algorithms. Integration of static and dynamic analysis will be 
desirable: verify what is practically feasible, and test (monitor) 
the remaining proof obligations.

\section{Conclusion}

We have in this document outlined some views on the potential in 
combining modeling and programming, supported by analysis 
capabilities such as static analysis, model checking, theorem 
proving, monitoring, and testing. We believe that the time is right 
for the formal methods/modeling and programming language 
communities to join forces. To some extent this is already 
happening in the small. However, we believe that we are standing in 
front of a major wave of research creating a united foundation for 
modeling, programming and verification. A cynical argument is that 
this is all obvious, which may be true. 

\section{References}

\subsection{Modeling}

\begin{itemize}
\item CIP: http://en.wikipedia.org/wiki/Wide-spectrum\_language
\item Coq: https://coq.inria.fr
\item JML: http://www.eecs.ucf.edu/~leavens/JML//index.shtml
\item Mathematica: http://www.wolfram.com/mathematica
\item Modelica: https://www.modelica.org
\item OBJ: http://c2.com/cgi/wiki?ObjLanguage
\item PVS: http://pvs.csl.sri.com
\item RAISE: http://spd-web.terma.com/Projects/RAISE
\item SysML: http://www.omgsysml.org
\item TLA: http://research.microsoft.com/en-us/um/people/lamport/tla/tla.html
\item UML: http://www.uml.org
\item VDM: http://en.wikipedia.org/wiki/Vienna\_Development\_Method
\item Z: http://en.wikipedia.org/wiki/Z\_notation
\end{itemize}

\subsection{Programming}

\begin{itemize}
\item Agda: http://wiki.portal.chalmers.se/agda/pmwiki.php
\item C: http://en.wikipedia.org/wiki/C\_(programming\_language)
\item C\#: https://msdn.microsoft.com/en-us/library/67ef8sbd.aspx
\item D: http://dlang.org
\item Dafny: http://research.microsoft.com/en-us/projects/dafny
\item Eiffel: https://www.eiffel.com
\item F\#: http://fsharp.org
\item Fortress: java.net/projects/projectfortress/pages/Home
\item Go: https://golang.org
\item Haskell: https://www.haskell.org
\item Java: https://www.oracle.com/java/index.html
\item JavaScript: http://www.w3schools.com/js/
\item LISP (CLOS): http://www.cs.cmu.edu/Groups/AI/html/cltl/cltl2.html
\item Ocaml: http://caml.inria.fr/ocaml
\item Python: https://www.python.org
\item Ruby: https://www.ruby-lang.org/en
\item RUST: http://www.rust-lang.org
\item Scala: http://www.scala-lang.org
\item SPARK: http://libre.adacore.com/tools/spark-gpl-edition
\item Spec\#: http://research.microsoft.com/en-us/projects/specsharp
\item Swift: https://developer.apple.com/swift
\end{itemize}

